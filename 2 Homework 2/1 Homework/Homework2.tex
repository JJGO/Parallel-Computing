\documentclass[a4paper]{article}
\input{plantilla.tex}

% Global variables
\newcommand{\tu}           {\underline}
\newcommand{\HWTitle}           {Homework 2}
\newcommand{\HWSubtitle}        {}
\newcommand{\HWDueDate}         {\today}
\newcommand{\HWClass}           {EECS 587 Parallel Computing}
\newcommand{\HWAuthorName}     {José Javier González Ortiz}

\title{\HWTitle \\ \vspace{.25cm} \large\HWSubtitle}
\author{\HWAuthorName}
\date{\HWDueDate}

\setlength{\parskip}{.5em}

\renewcommand\algorithmiccomment[1]{%
  \hfill\ \eqparbox{COMMENT}{#1}%
}

% \DeclarePairedDelimiter\par{\left(}{\right)}

\newcommand\LONGCOMMENT[1]{%
  \hfill\ \begin{minipage}[t]{\eqboxwidth{COMMENT}}#1\strut\end{minipage}%
}

\begin{document}
\maketitle
\thispagestyle{fancy}

\section{Efficiency Analysis}

    Given the Serial program we can analyze its Big-O notation. From the given information about initialize, statement 3, statement 5 and finalize we can calculate the serial time as follows:
    \begin{equation}
        \text{SerTime}(n) = \Theta(n) + n\cdot(\Theta(1) + n\Theta(1)) + \Theta(1) = \Theta(n^2)
    \end{equation}

    Since we can perfectly parallelize the inner loop the Parallel time without any communication will be:

    \begin{equation}
        \text{ParTime}(n,p) = \Theta(n) + n\cdot \left(\Theta(1) + \frac{n}{p}\Theta(1) \right) + \Theta(1) = \Theta \left(\frac{n^2}{p}\right) + \Theta(n)
    \end{equation}

    We cannot discard the $\Theta(n)$ term since for $p = \Theta(n)$ will as important as the first term. Introducing communication into the algorithm yields the following efficiencies:

    \begin{enumerate}[a)]
        \item Introducing a $\Theta(1)$ communication in the inner loop produces no changes in the big O representation of the Parallel Time
        
        \begin{equation}
            \text{ParTime}(n,p) = \Theta(n) + n\cdot \left(\Theta(1) + \frac{n}{p}\Theta(1) \right) + \Theta(1) = \Theta \left(\frac{n^2}{p}\right) + \Theta(n)
        \end{equation}

        The efficiency will be:

        \begin{equation}
            \text{Efficiency}(n,p) = \frac{\Theta(n^2)}{p \left( \Theta \left(\frac{n^2}{p}\right) + \Theta(n) \right)} = \frac{\Theta(n^2)}{\Theta(n^2)+  \Theta(n p)}
        \end{equation}

        This will only be constant as long as $ np = O(n^2)$, namely if $p = O(n)$. If $p = \omega(n)$ the efficiency will drop to zero, however we know by definition that $p \leq n$ since there cannot be more processors than iterations of the inner loop.

        \item Introducing a $\Theta(p)$ time in the outer loop the Parallel time will be:

        \begin{equation}
            \text{ParTime}(n,p) = \Theta(n) + n\cdot \left(\Theta(1) + \frac{n}{p}\Theta(1) + \Theta(p) \right) + \Theta(1) = \Theta \left(\frac{n^2}{p}\right) + \Theta(n p)
        \end{equation}

        With this expression the efficiency is:

        \begin{equation}
            \text{Efficiency}(n,p) = \frac{\Theta(n^2)}{p \left(\Theta \left(\frac{n^2}{p}\right) + \Theta(n p) \right)} = \frac{\Theta(n^2)}{\Theta(n^2)+  \Theta(n p^2)}
        \end{equation}

        As before, this ratio will tend to a constant given that $n p^2 = O(n^2)$. Simplifying the expression we get $p = O(\sqrt{n})$. For $p = \omega(\sqrt{n})$ the efficiency will go to zero.

        \item  Introducing a $\Theta(\sqrt{n/p})$ time in the outer loop the Parallel time will be:

        \begin{equation}
            \text{ParTime}(n,p) = \Theta(n) + n\cdot \left(\Theta(1) + \frac{n}{p}\Theta(1) + \Theta \left(\sqrt{\frac{n}{p}}\right) \right) + \Theta(1) = \Theta \left(\frac{n^2}{p}\right) + \Theta(n) + \Theta \left(n \sqrt{\frac{n}{p}}\right)
        \end{equation}

        The efficiency in this case follows a somewhat more complicate expression

        \begin{equation}
            \text{Efficiency}(n,p) = \frac{\Theta(n^2)}{p \left(\Theta \left(\frac{n^2}{p}\right) + \Theta(n) + \Theta \left(n \sqrt{\frac{n}{p}}\right) \right)} = \frac{\Theta(n^2)}{\Theta(n^2)+  \Theta(n p) + \Theta(\sqrt{n^3 p})}
        \end{equation}

        To get a better idea of the asymptotic behavior we can divide by $\Theta(n^2)$ and get

        \begin{equation}
            \text{Efficiency}(n,p) = \frac{\Theta(1)}{\Theta(1)+  \Theta \left(\dfrac{p}{n}\right) + \Theta \left(\sqrt{\dfrac{p}{n}}\right)}
        \end{equation}

        From this we have to satisfy both:
        \begin{equation}
           \dfrac{p}{n} = \Theta(1)  \qquad
           \sqrt{\dfrac{p}{n}} = \Theta(1)
        \end{equation}

        Which will only happen if $p = O(n)$. Then the efficiency will tend to a constant. As we said before $p \leq n$ so this will always be satisfied

    \end{enumerate}

\section{Reduce Sum}
    \begin{enumerate}[a)]
    \item We want to minimize the communication in order to improve the efficiency, so the best approach will be summing the $n/p$ values in each processor and then adding these intermediate sums. Since they are arranged in a 1-dimensional mesh the running time of the communication is $\Theta(p)$. The procedure would be something similar to the one described in Algorithm \ref{alg:RS}.
    \begin{algorithm}
        \caption{1D reduce(sum,A)}\label{alg:RS}
        \begin{algorithmic}[1]
            \Require{Each processor has the common values \emph{p} and \emph{n}, as well as the local value \emph{id}}
            \Statex
            \Procedure{Reduce(sum,A)}{}
                \State $s \gets sum[ A(id \cdot n/p:(id+1) \cdot n/p) $]
                \If{$id > 0$}
                    \State \emph{receive}($id-1$,$t$)
                    \State $s \gets s + t$
                \EndIf
                \If{$id < p-1$}
                    \State \emph{send}($id+1$,$s$)
                \EndIf
                \State \Return $s$
            \EndProcedure
        \end{algorithmic}
    \end{algorithm}

    \item As we can see from the previous question the algorithm will take $\Theta(n/p) + \Theta(p)$ where $1 \leq p\leq n$. It is easy to see that for both $p = \Theta(1)$ and $p = \Theta(n)$ the running time is $\Theta(n)$. We want to minimize so by making the terms being equal we will minimize the running time.
    \begin{equation}
        \Theta \left( \frac{n}{p} \right) = \Theta(p) \rightarrow p = \Theta(\sqrt{n})
    \end{equation}

    Therefore the best running time will happen when $p = \Theta(\sqrt{n})$ and will be $\text{ParTime} = \Theta(\sqrt{n})$

    
    \item In a fully connected computer, given a binary operator the best communication will be $\Theta(\log p)$. In this case the running time is going to be $\Theta(n/p) + \Theta(\log p)$.

    Since due to the communication constraint we know that there is a $\Omega(\log n)$ lower bound we can see that we cannot do better than getting rid of the linear term. This happens for $p = \Theta(n)$ which minimizes the time to $\text{ParTime} = \Theta(\log n)$
    
    \end{enumerate}

\section{SIMD Hypercube}
    \subsection{Analysis}
        As a first step is good to think about which is the minimum ideal runtime given the communication constraints of an hypercube. Let $p_M$ be the processor with the maximum value $v_M \in V : v_M \geq v_i \quad\forall v_i \in V$. There will exist a processor $\overline{p_M}$ such that the minimum distance from $p_M$ measured in edges will be $\log p$. To reason this, just think about this maximum point as a vertex in the hypercube and the processor $\overline{p_M}$ as the opposite vertex, since the hypercube has $\log p$ dimensions there is no shortest path between them. Furthermore, the coordinates $\overline{p_M}$ will be the bitwise complement of the coordinates of $\overline{p_M}$.

        Thus the algorithm will only be correct if $v_{\overline{p_M}} = v_M$ which will need at least $\Omega(\log p)$ communications. We have found a logarithmic lower bound in our running time. An efficient algorithm that solves the problem can be thought in the following manner:\\\,\\
        \centerline{
        {   In each timestep $i$ , each processor exchanges its value with its neighbor \phantom{a} }}\\\centerline{{ in the $i^{th}$ dimension and stores as its new value the maximum of the two.\phantom{}}}\\
        \centerline{ {After $\log p$ steps all the processors have the maximum value in the array.}
        }

        This algorithm is described in the Algorithm \ref{alg:SIMD}.

    \subsection{Algorithm}

        \begin{algorithm}
            \caption{SIMD Hypercube AllReduce(max)}\label{alg:SIMD}
            \begin{algorithmic}[1]
                \Require{Each processor has the common value $p$} and the local values \emph{pid} and $V$, as well as the variablew $M$ and $R$
                \Statex{\hspace{2.8em} The controller has the variables $dim$ and $x$ }
                \Statex
                \Procedure{AllReduce}{$\max,V(0:p-1)$}
                    \State $M \gets V$ \Comment{This instruction is transmitted and executed in the processors}
                    \State $x \gets \lg p$\Comment{This instruction is executed in the Controller}
                    \For{$dim = 0$ \textbf{to} $x-1$}\Comment{This instruction is executed in the Controller}
                        \State \emph{exchange}($dim,M,R$) \Comment{This instruction is transmitted and executed in the processors}
                        \If{$R > M$} \Comment{This instruction is transmitted and executed in the processors}
                            \State $M\gets R$ \Comment{This instruction is transmitted and executed in the processors}
                        \EndIf
                    \EndFor
                    \State \Return $M$ \Comment{This instruction is transmitted and executed in the processors}
                \EndProcedure
            \end{algorithmic}
        \end{algorithm}
\newpage
    \subsection{Correctness}
        For correctness first we state that given any processor $p$ it has only one neighbor in the dimension $dim$, namely $p \oplus 2^{dim}$.
        At each time step the number of different values in the array is at least halved because the processors are saving the maximum value and each processor is \emph{exchanging} the value with only one other processor. After $\log p$ steps we will have gone through all the dimensions and all the processors will have the maximum value $v_M$.

        Namely, let the $p_M$ be the processor with the maximum value $v_M \in V : v_M \geq v_i \quad\forall v_i \in V$, and be the bit representation of its \emph{pid} $b_{x-1}b_{x-2}\ldots b_2 b_1 b_0$. After each time step $dim$ the processors with \emph{pids} of the form shown in Equation \ref{eq:bit} have the maximum value because they have received it in any of the previous iterations. After $x = \log p$ steps all the processors will be within this set.
        \begin{equation}
            b_{x-1}b_{x-2}\ldots b_{dim+2} b_{dim+1} \underbrace{X \ldots X X}_{\text{$dim+1$ times}}.
            \label{eq:bit}
        \end{equation}

        We can also prove that given any processor $p$ there exists a chain of \emph{exchanges} less than $\log p$ that goes from $p_M$ to $p$.
            
    \subsection{Running Time}
        Finally, to calculate the running time we have to take into account that both the \emph{exchange} statement and the \emph{if} statement are $\Theta(1)$ because we are sending before receiving. Since there are $\log p$ steps the total running time will be $\Theta(\log p)$ which is the best we can do since there is a logarithmic lower bound as we saw due to the communication constraints.



\end{document}
